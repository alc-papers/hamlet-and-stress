\doublespacing{}
\noindent
Adrian Lucr\`{e}ce C\'{e}leste \\
Miss Thorpe \\
English 11 \\
November 13, 2015

\begin{center}
\underline{Hamlet} and Insanity
\end{center}
\paragraph{}
Stress can drive people to brash actions, whether it be the person suffering from
stress, or another person looking after the well-being of the stressed person.
\underline{Hamlet}, a dramatic play by William Shakespeare, establishes this
central idea, showing that even the simplest of stressors can drive one to
impetuous actions. Stress in aggregate with hasty actions often causes a person
to do regretable actions. Thoughtless actions fueled by stress and regret can
lead to missery.

\paragraph{}
Throughout \underline{Hamlet} there are multiple instances where a character in
the play does something reckless due to stress. For example in act 3 scene 4, in
a fit of rage Hamlet kills Polonius; ``Queen What will thou do? thou will not 
murder me? Help, help, ho! Polonius [Behind] What, ho! help, help, help! Hamlet 
[Drawing] How now!  a rat? Dead, for a ducat, dead! [Makes a pass through the
arras. Polonius [Behind] O, I am slain! [Falls and dies.'' (Act III, Scene 4).
In this scene Hamlet is talking to Gertrude (his mother and queen of Denmark)
trying to convince her that Cladius killed king Hamlet, though the queen panics
and Polonius gives up his location behind the arras and Hamlet kills him. The 
heat of the moment caused Hamlet not to think before he acted, resulting in the
death of Polonius. Another instance where a character in the play does something
because of stress is in act 3 scene 2 where the king runs out of the play
because Hamlet had they actors reinact the murder of King Hamlet, of which
Claudius is responsible for. ``Lucianus:\ldots{}Thou mixture rank, of midnight weeds
collelcted, With Hectate's ban thrice blasted, thrice infected, Thy natural 
magic and dire property, on wholesome life surp immediately [Pours the poison
into the sleeper's ear]~\ldots{}King: Give me some light. Away!'' (Act III, Scene 2).
In the reinactmeant of the murder in the play Lucianus is supposed to represent
Claudius, and the sleeper represents King Hamlet. Claudius feeling guilty for
killing King Hamlet and stressed by seeing the murder reinacted caused him to
run out of the room in panic. Stress can cause us to do things without thinking.

\paragraph{}
Reckless and hasty actions cause one to later regret what they have done. During
a sword fight between Hamlet and Laertes, Claudius spikes a goblet of wine with
poison, thinking Hamlet will drink it and die. Though this doesn't happen and
instead the queen ends up drinking it and dying. ``Queen: No, no, the drink, the
drink---O my dear Hamlet---The drink, the drink! I am poison'd'' (Act V, Scene 2)
in this moment the queen instead of listening Claudius she drank the poisoned
wine and died. Another instance a character does something they regret is when
Laertes and Hamlet are fencing and Laertes stabs hamlet with his poison tipped
rapier. ``Laertes: It is here, Hamlet: Hamlet, thou art slin; No medicine in the
world can do thee good, in thee there is not half an hour of life; The
treacherous instrument is in thy hand, Unbated and envenmod'd: the foul practice
Hath turn'd itself on me; lo, here I lie, Never to rise again: thy mother's
poison'd: I can no more: the kin, the king's to blame.'' (Act V, Scene 2).

\paragraph{}
So in conclusion, Stress can lead people to do foolhardy actions, whether it be
the stress person or, the people around them. In adittion Doing something hastily
can lead regretable actions. These two things in tandem can drive people to missery.
